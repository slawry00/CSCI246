\documentclass{article}
\usepackage{fasy-hw}

\author{Spencer Lawry}
\problem{1}
\collab{none}

\begin{document}

My photo in D2L has been updated to be a clearly identifiable photo

\problem{2}
\collab{none}
\clearpage
\header
LaTex template is being used here.

\problem{3}
\collab{none}
\clearpage
\header
\begin{enumerate}
\item Algorithms: I've learned some algorithms in my previous CS courses, mostly just sorting algorithms like quick sort, merge sort, insertion sort, etc. I do also remember the algorithm to rearrange a binary search tree.                                  
\item Data Structures: I've also learned quite a few data structures in my previous CS courses.This includes: lists, trees, stacks, hash tables, and queues. Of course there are multiple types of all of these data structures.                                                
\item Graphs: I've only been introduced to graphs for the most part. We just traversed a graph and the order at which you write the vertices depending on the strategy. It has been over a year since I did these though
\item Binomial Coefficients: Sounds vaguely familiar from Algebra 2 like 5 years ago. Sounds like I'll have to brush up on it a bit.
\item Proof by Counter-example: Self explanatory I assume. Using an example that contradicts a statement. No real experience with it though.
\item Proof by Example: Using something that shows the truth of a statement. This only works with statements that ask for a certain thing rather than it always being true.
\item Proof by Induction: My understanding of proof by induction is that you show that a statement is true for one case, then you extrapolate that to the nth case to show that it is true for all cases. I don't have much experience with this that I can remember.
\item Recursion in Code: I have a decent amount of experience from programming classes on how to use recursion in code. Calling the same function within itself and having a base case that will eventually be satisfied for an exit route keeps the program from running an endless loop.
\item Recurrence Relations: The term doesn't sound familiar, but the equations on the wikipedia page look familiar from past math classes, so I assume that I've used recurrence relations, but just didn't know the name previously. It is essentially a way to condense an equation that has a recursive pattern.
\item The Four Color Theorem: I don't have any experience with the Four Color Theorem outside of what was said in class about how many colors are needed to fill all of the countries on the map and not have the same color as their neighbors. 

\end{enumerate}

\problem{4}
\collab{none}
\clearpage
\header
I have reviewed all properties of real numbers in Appendix A.

\problem{5}
\collab{none}
\clearpage
\header
Section 4.1 Question 5: Epp, Susanna S. Discrete Mathematics: Introduction to Mathematical Reasoning (Page 125). Cengage Textbook. Kindle Edition. 
\begin{algorithm}
There are distinct integers m and n such that 1/m + 1/n is an integer. 
\end{algorithm}

Solution: Let m=2 and n=2. 1/2 + 1/2 = 1. 1 is an integer. Constructive Proof of Existence.

\problem{6}
\collab{none}
\clearpage
\header
Section 4.1 Question 46: Epp, Susanna S.. Discrete Mathematics: Introduction to Mathematical Reasoning (Page 126). Cengage Textbook. Kindle Edition. 

\begin{algorithm}The product of any even integer and any integer is even.
\end{algorithm}
Solution: Let "any even integer" be represented by 2(k) and "any integer" represented by m.
The product of these two integers will be 2(k)*m, which can be written as 2(k*m).
The product of two integers is always another integer, therefore 2(k*m) = 2*(m) which is the definition of an even number.
\problem{7}
\collab{none}
\clearpage
\header
Section 1.2 Question 5: Epp, Susanna S.. Discrete Mathematics: Introduction to Mathematical Reasoning (Page 13). Cengage Textbook. Kindle Edition. 
\begin{algorithm}
Which of the following sets are equal?
\begin{enumerate}

\item A = {0, 1, 2} 
\item B = {x∈R | −1≤x < 3} 
\item C = {x∈R | −1 < x < 3} 
\item D = {x∈Z| −1 < x < 3} 
\item E = {x∈Z+ | −1 < x < 3}

\end{enumerate}
\end{algorithm}

Solution: Only A and D are equal because D = the integers between -1 and 3 (non-inclusive), which is {0,1,2} and the same as A. The difference between B and C is that B includes -1, while C does not. The difference between E and D and A is that E doesn't include 0 because it technically isn't a positive integer.

\problem{8}
\collab{none}
\clearpage
\header
Section 1.3 Question 12: Epp, Susanna S.. Discrete Mathematics: Introduction to Mathematical Reasoning (Page 22). Cengage Textbook. Kindle Edition. 
\begin{algorithm}
Define a relation T from R to R as follows: For all real numbers x and y, (x, y)∈T means that y^2 − x^2 = 1. Is T a function? Explain.
\end{algorithm}


Solution: No, T is not a function because there are multiple solutions for Y for a single X. Say x is 1. Y is then either sqrt(2) or -sqrt(2). A function can only have 1 solution for both x and y.  

\problem{9}
\collab{none}
\clearpage
\header
Georg Cantor created set theory along with a ton of other important set functions and theories that changed mathematics entirely. Before he created his set theory, the previous set theories overlooked infinite sets as being something that can be proved or disproved with mathematics. \\
\newline
Sets are used extensively in computer science for data structures and in the study of logic and many other subjects. Without his work on sets, the future research and development to modern computers as well as computer programming could have taken much longer or potentially never happened at all. \newline

Sources:
\begin{enumerate}
\item “Georg Cantor,” Wikipedia, 31-Aug-2018. [Online]. Available: https://en.wikipedia.org/wiki/Georg_Cantor. [Accessed: 05-Sep-2018].
\end{enumerate}


 




\end{document}
