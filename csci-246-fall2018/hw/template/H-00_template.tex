\documentclass{article}
\usepackage{fasy-hw}

\author{Your Name Here}
\problem{1-1}
% \problem{A-B} means Problem Set A, Problem B.
\collab{none}
% or give names, e.g., \collab{Alyssa P. Hacker and A. Student}

\begin{document}

Your solution to Problem 1-1 goes here. Remember, \emph{each problem} should be
start on a new page.

If the problem has subparts, it will be convenient to use the enumerate
environment:
\begin{enumerate}[(a)]
    \item TODO: give the solution to subproblem (a) here.
    \item TODO: give the solution to subproblem (b) here.
\end{enumerate}

If for whatever reason enumerate will not suffice (perhaps if your answers are
getting rather long), then use subsection*:

\section*{Problem 1(a)}
TODO: give solution here.

This question requires you to include an algorithm.  Be sure to introduce the
algorithm first by describing the problem/input, then saying intuitively what
this algorithm does.  We solve this problem in \algref{coins}, where we [insert
short description here].

\begin{algorithm}
    \caption{Making Change}
    \label{alg:coins}
    \begin{algorithmic}
        \State \textbf{Input:} Integer $val$, array of integers $coins$.
        \State \textbf{Output:} The minimal number of coins needed to sum
        to exactly $val$ cents.\\
        \hrulefill
        \State ncoins $\gets$ array of length $val+1$
        \State ncoins$[0] \gets0$.
        \For{$ i = 1 \ldots val$}
        \State ncoins$[i]=\infty$.
        \For{$j=0 \ldots 5$}
        \State $k \gets i-coins[j]$
        \If{$ k \geq 0$}
        \State ncoins$[i] \gets \min\{\text{ncoins}[i],
        \text{ncoins}[k]+1\}$.
        \EndIf
        \EndFor
        \EndFor
        \State return ncoins$[val]$.
    \end{algorithmic}
\end{algorithm}


\section*{Problem 1(b)}
TODO: give solution here, etc.

\problem{1-2}
\collab{none}
\clearpage
\header

And the second problem should go on a fresh page.

\end{document}

